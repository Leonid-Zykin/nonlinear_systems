\section*{Заключение}

В данной лабораторной работе был проведен комплексный анализ семи нелинейных динамических систем. Работа включала в себя поиск точек равновесия, классификацию их типов, построение фазовых портретов и синтез стабилизирующих регуляторов.

\subsection*{Основные результаты анализа}

Для каждой из семи систем были найдены все точки равновесия и определены их типы:

\begin{itemize}
\item \textbf{Система 1} оказалась довольно интересной --- в ней обнаружились три точки равновесия: устойчивый фокус в начале координат и два седла в точках $(1,-1)$ и $(-1,1)$.

\item \textbf{Система 2} показала более сложную динамику с тремя точками равновесия: седлом в начале координат, неустойчивым узлом в точке $(0,1)$ и неустойчивым фокусом в точке $(1.26,-1)$.

\item \textbf{Система 3} имеет единственную точку равновесия --- неустойчивый фокус в начале координат.

\item \textbf{Система 4} оказалась особенно интересной --- помимо неустойчивого узла в начале координат, она имеет целую окружность точек равновесия, что привело к появлению устойчивого предельного цикла.

\item \textbf{Система 5} демонстрирует симметричную структуру с седлом в начале координат и двумя устойчивыми узлами в точках $(1,1)$ и $(-1,-1)$.

\item \textbf{Система 6} имеет вырожденную точку в начале координат и седло в точке $(1,1)$.

\item \textbf{Система 7} (трехмерная) оказалась вырожденной --- единственная точка равновесия $(0,0,0)$ имеет три нулевых собственных значения.
\end{itemize}

\subsection*{Особенности анализа}

Особое внимание было уделено анализу предельных циклов для систем 3 и 4. Переход к полярным координатам позволил получить уравнение $\dot{r} = r(1-r^2)$ для системы 4, что дало четкое понимание динамики: все траектории стремятся к устойчивому предельному циклу радиуса 1.

Для систем 1-6 были построены фазовые портреты, которые наглядно демонстрируют поведение траекторий в окрестности точек равновесия. Система 7 была исключена из визуализации согласно условиям задания.

\subsection*{Синтез регуляторов}

Особый интерес представляла задача синтеза стабилизирующих регуляторов для двух управляемых систем. Использование метода LQR позволило получить эффективные регуляторы, которые успешно стабилизируют системы в окрестности выбранных точек равновесия.

\subsection*{Практические выводы}

Проведенная работа показала важность комбинированного подхода к анализу нелинейных систем --- сочетание аналитических методов с численными расчетами дает наиболее полную картину динамического поведения.

Полученные результаты могут быть полезны при проектировании систем управления для реальных технических объектов, где нелинейность играет существенную роль.

\textbf{В заключение} можно отметить, что методы анализа нелинейных систем, рассмотренные в работе, являются мощным инструментом для понимания сложной динамики и позволяют эффективно решать задачи стабилизации.
