\section*{Введение}

В данной лабораторной работе рассматриваются методы анализа нелинейных систем управления. Основными задачами являются:

\begin{enumerate}
\item Поиск точек равновесия нелинейных систем
\item Определение типа точек равновесия методом линеаризации
\item Анализ устойчивости предельного цикла (только для системы 4)
\item Синтез стабилизирующих регуляторов
\end{enumerate}

\subsection*{Теоретические основы}

\subsubsection*{Точки равновесия}

Точкой равновесия нелинейной системы $\dot{\mathbf{x}} = \mathbf{f}(\mathbf{x})$ называется такое состояние $\mathbf{x}^*$, при котором $\mathbf{f}(\mathbf{x}^*) = \mathbf{0}$.

Для анализа устойчивости точки равновесия используется метод линеаризации. Матрица Якоби системы вычисляется как:
$$J_{ij} = \frac{\partial f_i}{\partial x_j}\bigg|_{\mathbf{x} = \mathbf{x}^*}$$

Тип точки равновесия определяется собственными значениями матрицы Якоби:
\begin{itemize}
\item Все $\text{Re}(\lambda_i) < 0$ --- устойчивая точка
\item Есть $\text{Re}(\lambda_i) > 0$ --- неустойчивая точка
\item $\text{Re}(\lambda_i) = 0$ --- критический случай
\end{itemize}

\subsubsection*{Классификация точек равновесия}

В зависимости от собственных значений матрицы Якоби различают следующие типы точек равновесия:

\textbf{Узлы:}
\begin{itemize}
\item Устойчивый узел: все $\lambda_i$ действительные и отрицательные
\item Неустойчивый узел: все $\lambda_i$ действительные и положительные
\end{itemize}

\textbf{Фокусы:}
\begin{itemize}
\item Устойчивый фокус: $\lambda_i$ комплексные с отрицательной действительной частью
\item Неустойчивый фокус: $\lambda_i$ комплексные с положительной действительной частью
\end{itemize}

\textbf{Седла:} есть как положительные, так и отрицательные действительные части собственных значений

\textbf{Центры:} $\lambda_i$ чисто мнимые (действительная часть равна нулю)

\subsubsection*{Предельные циклы}

Предельным циклом называется изолированная замкнутая траектория в фазовом пространстве. Для анализа предельных циклов часто используется переход к полярным координатам.

\subsubsection*{Стабилизация систем}

Для стабилизации нелинейных систем в окрестности точки равновесия применяются методы синтеза регуляторов, в частности, LQR (Linear-Quadratic Regulator) метод.

LQR регулятор минимизирует функционал:
$$J = \int_0^{\infty} (\mathbf{x}^T Q \mathbf{x} + \mathbf{u}^T R \mathbf{u}) dt$$

где $Q$ и $R$ --- матрицы весов.

Решение задачи LQR сводится к решению алгебраического уравнения Риккати:
$$A^T P + PA - PBR^{-1}B^T P + Q = 0$$

Матрица обратной связи вычисляется как:
$$K = R^{-1}B^T P$$
