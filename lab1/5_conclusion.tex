\section*{Заключение}

В ходе выполнения лабораторной работы были решены следующие задачи:

\begin{enumerate}
\item \textbf{Анализ точек равновесия:} Для всех семи систем найдены точки равновесия аналитическими и численными методами.

\item \textbf{Классификация точек равновесия:} С использованием метода линеаризации определены типы всех изолированных точек равновесия:
\begin{itemize}
\item Система 1: устойчивый фокус $(0,0)$ и два седла $(1,-1)$, $(-1,1)$
\item Система 2: седло $(0,0)$, неустойчивый узел $(0,1)$ и седло $(1.26,-1)$
\item Система 3: неустойчивый фокус $(0,0)$
\item Система 4: изолированная точка $(0,0)$ и континуум точек на окружности
\item Система 5: седло $(0,0)$ и два устойчивых узла $(1,1)$, $(-1,-1)$
\item Система 6: вырожденная точка $(0,0)$ и седло $(1,1)$
\item Система 7: неустойчивый фокус $(0.739, 0.739, -0.123)$
\end{itemize}

\item \textbf{Фазовые портреты:} Построены численные фазовые портреты для всех систем, включая 3D визуализацию для системы 7.

\item \textbf{Анализ предельных циклов:} Для систем 3 и 4 проведен анализ предельных циклов с использованием полярных координат.

\item \textbf{Синтез регуляторов:} Для двух управляемых систем синтезированы LQR регуляторы и проведено численное моделирование их работы.
\end{enumerate}

\textbf{Основные результаты:}

\begin{itemize}
\item Разработаны эффективные алгоритмы поиска точек равновесия, сочетающие аналитические и численные методы
\item Показана эффективность LQR регуляторов для стабилизации нелинейных систем в окрестности точек равновесия
\item Получены качественные и количественные характеристики динамики исследуемых систем
\end{itemize}

\textbf{Практическая значимость:} Результаты работы могут быть использованы при проектировании систем управления для стабилизации нелинейных объектов в окрестности заданных точек равновесия.

\textbf{Выводы:} Методы анализа нелинейных систем, рассмотренные в работе, позволяют эффективно исследовать динамические свойства сложных систем и синтезировать стабилизирующие регуляторы.
