\textbf{Единственная точка равновесия:} $(0, 0, 0)$

Проверка:
\begin{itemize}
\item $f_1(0,0,0) = -0^3 + 0^3 = 0$ ✓
\item $f_2(0,0,0) = 0 + 3 \cdot 0 - 0^3 = 0$ ✓
\item $f_3(0,0,0) = 0 \cdot 0 - 0^3 - \sin(0) = 0$ ✓
\end{itemize}

\textbf{Матрица Якоби:}
$$J = \begin{pmatrix} 
-3x_1^2 & 3x_2^2 & 0 \\
1 & -3x_2^2 & 3 \\
x_3 - \cos x_1 & -3x_2^2 & x_1
\end{pmatrix}$$

В точке $(0, 0, 0)$:
$$J = \begin{pmatrix} 
0 & 0 & 0 \\
1 & 0 & 3 \\
-1 & 0 & 0
\end{pmatrix}$$

\textbf{Собственные значения:} $\lambda_1 = \lambda_2 = \lambda_3 = 0$ (тройной корень)

\textbf{Тип точки:} Вырожденный случай - требуется нелинейный анализ
