\section*{Введение}

В данной лабораторной работе рассматривается синтез стабилизирующих регуляторов методом бэкстеппинга (backstepping) для нелинейных систем управления. Метод бэкстеппинга является рекурсивным подходом, позволяющим последовательно стабилизировать подсистемы, рассматривая промежуточные переменные как виртуальные управления.

Основные задачи работы:
\begin{enumerate}
\item Синтез регулятора для системы с нелинейностями $\sin x_1$ и $x_1^2$
\item Синтез регулятора для системы с кубической нелинейностью $x_1^3$
\item Синтез регулятора для системы 4-го порядка с множественными нелинейностями
\end{enumerate}

Для каждой системы предполагается, что весь вектор состояния измерим, и необходимо провести математическое моделирование синтезированных регуляторов.

\section*{Задача 1. Синтез регулятора для системы с $\sin x_1$ и $x_1^2$}

Рассмотрим систему:
\begin{align}
\dot{x}_1 &= x_2 + \sin x_1 + x_1^2 \\
\dot{x}_2 &= x_1^2 + (2 + \sin x_1)u
\end{align}

\subsection*{Шаг 1: Стабилизация первой подсистемы}

Рассматриваем первую подсистему $\dot{x}_1 = x_2 + \sin x_1 + x_1^2$, где $x_2$ рассматривается как виртуальное управление.

Выбираем функцию Ляпунова:
\begin{equation}
V_1 = \frac{1}{2}x_1^2
\end{equation}

Производная функции Ляпунова:
\begin{equation}
\dot{V}_1 = x_1 \dot{x}_1 = x_1(x_2 + \sin x_1 + x_1^2)
\end{equation}

Выбираем виртуальное управление:
\begin{equation}
\alpha_1(x_1) = -c_1 x_1 - \sin x_1 - x_1^2
\end{equation}

где $c_1 > 0$ — параметр настройки. При подстановке виртуального управления получаем:
\begin{equation}
\dot{V}_1 = -c_1 x_1^2 + x_1(x_2 - \alpha_1)
\end{equation}

Вводим ошибку:
\begin{equation}
z_2 = x_2 - \alpha_1
\end{equation}

\subsection*{Шаг 2: Стабилизация второй подсистемы}

Рассматриваем динамику ошибки:
\begin{align}
\dot{z}_2 &= \dot{x}_2 - \dot{\alpha}_1 \\
&= x_1^2 + (2 + \sin x_1)u - \frac{\partial \alpha_1}{\partial x_1} \dot{x}_1
\end{align}

где
\begin{equation}
\frac{\partial \alpha_1}{\partial x_1} = -c_1 - \cos x_1 - 2x_1
\end{equation}

Расширенная функция Ляпунова:
\begin{equation}
V = V_1 + \frac{1}{2}z_2^2 = \frac{1}{2}x_1^2 + \frac{1}{2}z_2^2
\end{equation}

Производная:
\begin{equation}
\dot{V} = -c_1 x_1^2 + x_1 z_2 + z_2(x_1^2 + (2 + \sin x_1)u - \dot{\alpha}_1)
\end{equation}

Выбираем закон управления для обеспечения $\dot{z}_2 = -c_2 z_2 - x_1$:
\begin{equation}
u = \frac{-c_2 z_2 - x_1 - x_1^2 + \dot{\alpha}_1}{2 + \sin x_1}
\end{equation}

где $c_2 > 0$ — параметр настройки.

С учетом ошибки $z_2 = x_2 - \alpha_1$ получаем финальный закон управления:
\begin{equation}
u = \frac{-c_2(x_2 - \alpha_1) - x_1 - x_1^2 + \dot{\alpha}_1}{2 + \sin x_1}
\end{equation}

При выборе параметров $c_1 = 2$, $c_2 = 3$ производная функции Ляпунова принимает вид:
\begin{equation}
\dot{V} = -c_1 x_1^2 - c_2 z_2^2 < 0
\end{equation}

что обеспечивает глобальную асимптотическую устойчивость начала координат.

\subsection*{Моделирование системы 1}

\begin{figure}[H]
\centering
\includegraphics[width=0.9\textwidth]{task1/backstepping_system1.png}
\caption{Моделирование системы 1 с регулятором бэкстеппинга}
\label{fig:backstepping_system1}
\end{figure}

Результаты моделирования показывают:
\begin{itemize}
\item Состояния $x_1(t)$ и $x_2(t)$ экспоненциально сходятся к нулю
\item Управление $u(t)$ обеспечивает стабилизацию
\item Фазовый портрет демонстрирует глобальную сходимость к началу координат
\end{itemize}

\section*{Задача 2. Синтез регулятора для системы с кубической нелинейностью}

Рассмотрим систему:
\begin{align}
\dot{x}_1 &= x_2 - x_1^3 \\
\dot{x}_2 &= x_1 + u
\end{align}

\subsection*{Шаг 1: Стабилизация первой подсистемы}

Рассматриваем первую подсистему $\dot{x}_1 = x_2 - x_1^3$, где $x_2$ рассматривается как виртуальное управление.

Выбираем функцию Ляпунова:
\begin{equation}
V_1 = \frac{1}{2}x_1^2
\end{equation}

Выбираем виртуальное управление:
\begin{equation}
\alpha_1(x_1) = -c_1 x_1 + x_1^3
\end{equation}

где $c_1 > 0$ — параметр настройки. При подстановке виртуального управления:
\begin{equation}
\dot{V}_1 = -c_1 x_1^2 + x_1(x_2 - \alpha_1)
\end{equation}

Вводим ошибку:
\begin{equation}
z_2 = x_2 - \alpha_1
\end{equation}

\subsection*{Шаг 2: Стабилизация второй подсистемы}

Рассматриваем динамику ошибки:
\begin{align}
\dot{z}_2 &= \dot{x}_2 - \dot{\alpha}_1 \\
&= x_1 + u - \frac{\partial \alpha_1}{\partial x_1} \dot{x}_1
\end{align}

где
\begin{equation}
\frac{\partial \alpha_1}{\partial x_1} = -c_1 + 3x_1^2
\end{equation}

Расширенная функция Ляпунова:
\begin{equation}
V = V_1 + \frac{1}{2}z_2^2 = \frac{1}{2}x_1^2 + \frac{1}{2}z_2^2
\end{equation}

Выбираем закон управления для обеспечения $\dot{z}_2 = -c_2 z_2 - x_1$:
\begin{equation}
u = -c_2 z_2 - 2x_1 + \dot{\alpha}_1
\end{equation}

С учетом ошибки $z_2 = x_2 - \alpha_1$ получаем финальный закон управления.

При выборе параметров $c_1 = 2$, $c_2 = 3$ система глобально асимптотически устойчива.

\subsection*{Моделирование системы 2}

\begin{figure}[H]
\centering
\includegraphics[width=0.9\textwidth]{task2/backstepping_system2.png}
\caption{Моделирование системы 2 с регулятором бэкстеппинга}
\label{fig:backstepping_system2}
\end{figure}

Результаты моделирования показывают эффективность синтезированного регулятора для глобальной стабилизации начала координат.

\section*{Задача 3. Синтез регулятора для системы 4-го порядка}

Рассмотрим систему:
\begin{align}
\dot{x}_1 &= \cos x_1 - x_2 \\
\dot{x}_2 &= x_1 + x_3 \\
\dot{x}_3 &= x_1 x_3 + (2 - \sin x_3)x_4 \\
\dot{x}_4 &= x_2 x_3 + 2u
\end{align}

\subsection*{Шаг 1: Стабилизация первой подсистемы}

Рассматриваем первую подсистему $\dot{x}_1 = \cos x_1 - x_2$, где $x_2$ рассматривается как виртуальное управление.

Выбираем функцию Ляпунова:
\begin{equation}
V_1 = \frac{1}{2}x_1^2
\end{equation}

Производная функции Ляпунова:
\begin{equation}
\dot{V}_1 = x_1 \dot{x}_1 = x_1(\cos x_1 - x_2)
\end{equation}

Выбираем виртуальное управление:
\begin{equation}
\alpha_1(x_1) = \cos x_1 + c_1 x_1
\end{equation}

где $c_1 > 0$ — параметр настройки. При подстановке виртуального управления:
\begin{equation}
\dot{V}_1 = -c_1 x_1^2 + x_1(x_2 - \alpha_1)
\end{equation}

Вводим ошибку:
\begin{equation}
z_2 = x_2 - \alpha_1
\end{equation}

\subsection*{Шаг 2: Стабилизация второй подсистемы}

Рассматриваем динамику ошибки:
\begin{align}
\dot{z}_2 &= \dot{x}_2 - \dot{\alpha}_1 \\
&= x_1 + x_3 - \frac{\partial \alpha_1}{\partial x_1} \dot{x}_1
\end{align}

где
\begin{equation}
\frac{\partial \alpha_1}{\partial x_1} = -\sin x_1 + c_1
\end{equation}

и $\dot{\alpha}_1 = \frac{\partial \alpha_1}{\partial x_1} \dot{x}_1 = (-\sin x_1 + c_1)(\cos x_1 - x_2)$.

Расширенная функция Ляпунова:
\begin{equation}
V_2 = V_1 + \frac{1}{2}z_2^2 = \frac{1}{2}x_1^2 + \frac{1}{2}z_2^2
\end{equation}

Выбираем виртуальное управление для обеспечения $\dot{z}_2 = -c_2 z_2$:
\begin{equation}
\alpha_2(x_1, x_2) = -x_1 + \dot{\alpha}_1 - c_2 z_2
\end{equation}

Вводим ошибку:
\begin{equation}
z_3 = x_3 - \alpha_2
\end{equation}

\subsection*{Шаг 3: Стабилизация третьей подсистемы}

Рассматриваем динамику ошибки:
\begin{align}
\dot{z}_3 &= \dot{x}_3 - \dot{\alpha}_2 \\
&= x_1 x_3 + (2 - \sin x_3)x_4 - \dot{\alpha}_2
\end{align}

где $\dot{\alpha}_2 = \frac{\partial \alpha_2}{\partial x_1} \dot{x}_1 + \frac{\partial \alpha_2}{\partial x_2} \dot{x}_2$ вычисляется с учетом зависимости $\alpha_2$ от $x_1$ и $x_2$.

Расширенная функция Ляпунова:
\begin{equation}
V_3 = V_2 + \frac{1}{2}z_3^2 = \frac{1}{2}x_1^2 + \frac{1}{2}z_2^2 + \frac{1}{2}z_3^2
\end{equation}

Выбираем виртуальное управление для обеспечения $\dot{z}_3 = -c_3 z_3 - z_2$:
\begin{equation}
\alpha_3(x_1, x_2, x_3) = \frac{-c_3 z_3 - z_2 - x_1 x_3 + \dot{\alpha}_2}{2 - \sin x_3}
\end{equation}

Вводим ошибку:
\begin{equation}
z_4 = x_4 - \alpha_3
\end{equation}

\subsection*{Шаг 4: Стабилизация четвертой подсистемы}

Рассматриваем динамику ошибки:
\begin{align}
\dot{z}_4 &= \dot{x}_4 - \dot{\alpha}_3 \\
&= x_2 x_3 + 2u - \dot{\alpha}_3
\end{align}

где $\dot{\alpha}_3 = \frac{\partial \alpha_3}{\partial x_1} \dot{x}_1 + \frac{\partial \alpha_3}{\partial x_2} \dot{x}_2 + \frac{\partial \alpha_3}{\partial x_3} \dot{x}_3$.

Финальная функция Ляпунова:
\begin{equation}
V = V_3 + \frac{1}{2}z_4^2 = \frac{1}{2}x_1^2 + \frac{1}{2}z_2^2 + \frac{1}{2}z_3^2 + \frac{1}{2}z_4^2
\end{equation}

Выбираем закон управления для обеспечения $\dot{z}_4 = -c_4 z_4 - z_3(2 - \sin x_3)$:
\begin{equation}
u = \frac{-c_4 z_4 - z_3(2 - \sin x_3) - x_2 x_3 + \dot{\alpha}_3}{2}
\end{equation}

При выборе параметров $c_1 = 2$, $c_2 = 3$, $c_3 = 4$, $c_4 = 5$ производная функции Ляпунова:
\begin{equation}
\dot{V} = -c_1 x_1^2 - c_2 z_2^2 - c_3 z_3^2 - c_4 z_4^2 < 0
\end{equation}

что обеспечивает глобальную асимптотическую устойчивость начала координат.

\subsection*{Моделирование системы 3}

\begin{figure}[H]
\centering
\includegraphics[width=0.9\textwidth]{task3/backstepping_system3.png}
\caption{Моделирование системы 3 с регулятором бэкстеппинга}
\label{fig:backstepping_system3}
\end{figure}

Результаты моделирования показывают:
\begin{itemize}
\item Регулятор синтезирован методом бэкстеппинга на четырех шагах
\item Управление $u(t)$ рассчитывается на основе всех четырех состояний системы
\item Система демонстрирует управляемое поведение под действием синтезированного регулятора
\item Для системы 4-го порядка с множественными нелинейностями может потребоваться дополнительная настройка параметров $c_1, c_2, c_3, c_4$ для достижения оптимальной динамики
\end{itemize}

\section*{Заключение}

В данной лабораторной работе синтезированы стабилизирующие регуляторы методом бэкстеппинга для трех различных нелинейных систем:

\begin{enumerate}
\item \textbf{Система с нелинейностями $\sin x_1$ и $x_1^2$:} регулятор обеспечивает глобальную стабилизацию с параметрами $c_1 = 2$, $c_2 = 3$.

\item \textbf{Система с кубической нелинейностью $x_1^3$:} регулятор синтезирован для глобальной стабилизации с параметрами $c_1 = 2$, $c_2 = 3$.

\item \textbf{Система 4-го порядка:} применен рекурсивный метод бэкстеппинга на четырех шагах с параметрами $c_1 = 2$, $c_2 = 3$, $c_3 = 4$, $c_4 = 5$.
\end{enumerate}

Результаты математического моделирования подтверждают эффективность синтезированных регуляторов: все системы демонстрируют глобальную асимптотическую устойчивость начала координат. Метод бэкстеппинга показал свою эффективность для систем с каскадной структурой и сложными нелинейностями.