\section*{Введение}

В данной лабораторной работе рассматриваются методы линеаризации обратной связью для нелинейных систем управления. Основное внимание уделяется анализу линеаризуемости по входу-выходу, преобразованию систем в нормальную форму и синтезу законов управления.

Основные задачи работы:
\begin{enumerate}
\item Анализ линеаризуемости по входу-выходу нелинейной системы
\item Преобразование системы в нормальную форму с указанием области определения
\item Проверка минимально-фазовости системы
\item Синтез закона управления методом линеаризации обратной связью для глобальной стабилизации
\end{enumerate}

Работа демонстрирует применение теоретических методов линеаризации обратной связью к практическим задачам управления нелинейными системами.

\section*{Задача 1. Анализ линеаризуемости по входу-выходу}

Рассмотрим систему:
\begin{align}
\dot{x}_1 &= -x_1 + x_2 - x_3 \\
\dot{x}_2 &= -x_1 x_3 - x_2 + u \\
\dot{x}_3 &= -x_1 + u \\
y &= x_3
\end{align}

\subsection*{Проверка линеаризуемости по входу-выходу}

Для проверки линеаризуемости по входу-выходу вычислим производные Ли выходной функции $h(x) = x_3$.

Рассчитаем первую производную $y$:

\begin{align}
  \dot{y} = L_f h + L_g h \cdot u
\end{align}

\textbf{Шаг 1:} Вычисление производных Ли
% производная на саму функцию, смотрим есть ли управление в результате
\begin{align}
L_f^0 h &= h = x_3 \\
L_f^1 h &= \frac{\partial h}{\partial x_1} f_1 + \frac{\partial h}{\partial x_2} f_2 + \frac{\partial h}{\partial x_3} f_3 \\
&= 0 \cdot (-x_1 + x_2 - x_3) + 0 \cdot (-x_1 x_3 - x_2) + 1 \cdot (-x_1) \\
&= -x_1
\end{align}

\textbf{Шаг 2:} Проверка условия линеаризуемости
% g - коэффиценты при u в системе
\begin{align}
L_g L_f^0 h &= \frac{\partial h}{\partial x_1} g_1 + \frac{\partial h}{\partial x_2} g_2 + \frac{\partial h}{\partial x_3} g_3 \\
&= 0 \cdot 0 + 0 \cdot 1 + 1 \cdot 1 = 1 \neq 0
\end{align}

Таким образом получаем:

\begin{align}
  \dot{y} = L_f h + L_g h \cdot u = -x_1 + 1 \cdot u = -x_1 + u
\end{align}

% Относительная степень ( r ) — это сколько раз нужно продифференцировать выход ( y ) по времени, 
% чтобы управление ( u ) впервые явно появилось в выражении.
Поскольку $L_g L_f^0 h = 1 \neq 0$, система линеаризуема по входу-выходу с относительной степенью $r = 1$, так как вход 
$u$ появляется уже в первой производной выхода.

\subsection*{Преобразование в нормальную форму}

Для системы размерности $n = 3$ с относительной степенью $r = 1$ размерность внутренней динамики равна $n - r = 2$.

\textbf{Координаты нормальной формы:}
\begin{align}
z_1 &= h = x_3 \\
z_2 &= L_f h = -x_1 + u
\end{align}

% ( x_1 ) и ( x_2 ) называются внутренними координатами,
% потому что они не входят напрямую в выход ( y = x_3 )
\textbf{Внутренние координаты:}
\begin{align}
\eta_1 &= x_1 \\
\eta_2 &= x_2
\end{align}

\textbf{Производные координат нормальной формы:}
\begin{align}
\dot{z}_1 &= \dot{x}_3 = -x_1 + u \\
\dot{z}_2 &= \frac{d}{dt}(L_f h) = \frac{d}{dt}(-x_1 + u) = -\dot{x}_1 + \dot{u} \\
&= -(-x_1 + x_2 - x_3) + \dot{u} = x_1 - x_2 + x_3 + \dot{u}
\end{align}

\textbf{Область определения преобразования:}
Преобразование определено для всех $x \in \mathbb{R}^3$. Обратное преобразование:
\begin{align}
x_1 &= \eta_1 \\
x_2 &= \eta_2 \\
x_3 &= z_1
\end{align}

\subsection*{Проверка минимально-фазовости}

Для проверки минимально-фазовости анализируем внутреннюю динамику при нулевом выходе $y = z_1 = 0$ и $\dot{y} = z_2 = 0$.

При $z_1 = 0$ имеем $x_3 = 0$. 

При $z_2 = 0$ имеем $-x_1 + u = 0 \Rightarrow u = x_1$.

Внутренняя динамика:
\begin{align}
\dot{\eta}_1 = \dot{x}_1 &= -x_1 + x_2 \\
\dot{\eta}_2 = \dot{x}_2 &= -x_1 \cdot 0 - x_2 + u = -x_2 + u
\end{align}

При $u = x_1$:
\begin{align}
\dot{x}_1 &= -x_1 + x_2 \\
\dot{x}_2 &= x_1 - x_2
\end{align}

Матрица линеаризации внутренней динамики:
\begin{equation}
A = \begin{pmatrix} -1 & 1 \\ 1 & -1 \end{pmatrix}
\end{equation}

Собственные значения: $\lambda_1 = 0$, $\lambda_2 = -2$.

Поскольку один собственный корень равен нулю (неотрицательная вещественная часть не является строго отрицательной), 
нулевая динамика не асимптотически устойчива, следовательно система не минимально-фазовая.

% Минимально-фазовость означает, что внутренняя динамика системы 
% (та часть, которая не видна через выход) устойчива.
% Что это даёт:
% 1. Если система минимально-фазовая, то при управлении выходом внутренние переменные 
% не будут «разбегаться» или становиться неустойчивыми.
% 2. Это важно для обратной связи: можно стабилизировать всю систему, управляя только выходом.
% 3. Если система не минимально-фазовая, то даже при хорошем управлении выходом внутренние 
% координаты могут вести себя плохо (например, расти бесконечно).


% Раздел моделирования для задачи 1 исключён согласно требованиям задания (изображения не требуются)

\subsection*{Результаты задачи 1}

\textbf{Ответы:}
\begin{enumerate}
\item \textbf{Линеаризуемость:} Да, система линеаризуема по входу-выходу
\item \textbf{Относительная степень:} $r = 1$
\item \textbf{Нормальная форма:} получена с координатами $z_1 = x_3$, $z_2 = -x_1 + u$, $\eta_1 = x_1$, $\eta_2 = x_2$
\item \textbf{Область определения:} $\mathbb{R}^3$
\item \textbf{Минимально-фазовость:} Система не минимально-фазовая
\end{enumerate}

\section*{Задача 2. Синтез закона управления методом линеаризации обратной связью}

Рассмотрим систему:
\begin{align}
\dot{x}_1 &= -x_1 + x_2 \\
\dot{x}_2 &= x_1 - x_2 - x_1 x_3 + u \\
\dot{x}_3 &= x_1 + x_1 x_2 - 2x_3
\end{align}

Требуется найти закон управления с обратной связью по состоянию, обеспечивающий глобальную стабилизацию начала координат.

\subsection*{Анализ управляемости}

Проверим управляемость системы через скобки Ли.

Векторное поле $g = [0, 1, 0]^T$ (коэффициенты при $u$).

Скобка Ли $[f, g] = L_f g - L_g f$. Так как $g$ постоянно, $L_f g = 0$, и
\begin{equation*}
[f,g] = -\frac{\partial f}{\partial x} \cdot g = -\begin{pmatrix}
-1 & 1 & 0 \\
 1 - x_3 & -1 & -x_1 \\
 1 + x_2 & x_1 & -2
\end{pmatrix}\,\begin{pmatrix}0\\1\\0\end{pmatrix} = \begin{pmatrix}-1\\1\\-x_1\end{pmatrix}.
\end{equation*}

Далее $[f,[f,g]] = L_f [f,g] - L_{[f,g]} f$. В начале координат $x=0$ имеем
\begin{equation*}
g(0) = \begin{pmatrix}0\\1\\0\end{pmatrix},\quad [f,g](0)=\begin{pmatrix}-1\\1\\0\end{pmatrix},\quad [f,[f,g]](0)=\begin{pmatrix}-2\\2\\1\end{pmatrix}.
\end{equation*}

Матрица управляемости (столбцы $g,\ [f,g],\ [f,[f,g]]$) в начале координат:
\begin{equation}
C = \begin{pmatrix}
0 & -1 & -2 \\
1 &  1 &  2 \\
0 &  0 &  1
\end{pmatrix}
\end{equation}

Ранг матрицы управляемости равен 3, что совпадает с размерностью системы. Система локально управляема в начале координат.

\subsection*{Проектирование регулятора}

Выберем выходную функцию $h(x) = x_1$ и применим метод линеаризации обратной связью.

\textbf{Шаг 1:} Вычисление производных Ли
\begin{align}
L_f h &= \frac{\partial h}{\partial x_1} f_1 + \frac{\partial h}{\partial x_2} f_2 + \frac{\partial h}{\partial x_3} f_3 \\
&= 1 \cdot (-x_1 + x_2) + 0 \cdot (x_1 - x_2 - x_1 x_3) + 0 \cdot (x_1 + x_1 x_2 - 2x_3) \\
&= -x_1 + x_2
\end{align}

\begin{align}
L_g h &= \frac{\partial (h)}{\partial x_1} g_1 + \frac{\partial (h)}{\partial x_2} g_2 + \frac{\partial (h)}{\partial x_3} g_3 \\
&= 1 \cdot 0 + 0 \cdot 1 + 0 \cdot 0 = 0
\end{align}

\begin{align}
L_g L_f h &= \frac{\partial (L_f h)}{\partial x_1} g_1 + \frac{\partial (L_f h)}{\partial x_2} g_2 + \frac{\partial (L_f h)}{\partial x_3} g_3 \\
&= (-1) \cdot 0 + 1 \cdot 1 + 0 \cdot 0 = 1 \neq 0
\end{align}

% только во второй производной появляется управление
% x_1' = -x_1 + x_2
% (-x_1 + x_2)' = -x_1' + x_2' = -(-x_1 + x_2) + (x_1 - x_2 - x_1 x_3 + u) = u - x_1 x_3
Относительная степень $r = 2$, так как $u$ появляется во второй производной выходной функции.


\textbf{Шаг 2:} Синтез закона управления

Координаты нормальной формы:
\begin{align}
z_1 &= h = x_1 \\
z_2 &= L_f h = -x_1 + x_2
\end{align}

Вычисляем $L_f^2 h$:
\begin{align}
L_f^2 h &= \frac{\partial (L_f h)}{\partial x_1} f_1 + \frac{\partial (L_f h)}{\partial x_2} f_2 + \frac{\partial (L_f h)}{\partial x_3} f_3 \\
&= (-1) \cdot (-x_1 + x_2) + 1 \cdot (x_1 - x_2 - x_1 x_3) + 0 \cdot (x_1 + x_1 x_2 - 2x_3) \\
&= x_1 - x_2 + x_1 - x_2 - x_1 x_3 = 2x_1 - 2x_2 - x_1 x_3
\end{align}

Закон управления:
\begin{equation}
u = \frac{v - L_f^2 h}{L_g L_f h} = \frac{v - (2x_1 - 2x_2 - x_1 x_3)}{1} = v - 2x_1 + 2x_2 + x_1 x_3
\end{equation}

Выбираем $v = -k_1 z_1 - k_2 z_2 = -k_1 x_1 - k_2 (-x_1 + x_2)$ для стабилизации.

При $k_1 = 2$, $k_2 = 3$:
\begin{equation}
u = -2x_1 - 3(-x_1 + x_2) - 2x_1 + 2x_2 + x_1 x_3 = -x_1 - x_2 + x_1 x_3
\end{equation}

\subsection*{Линеаризация внешней составляющей и нуль-динамика}

Внешние координаты нормальной формы:
\begin{equation*}
z_1 = h(x) = x_1,\qquad z_2 = L_f h(x) = -x_1 + x_2.
\end{equation*}
Для выбранного управления получаем линейную внешнюю динамику
\begin{equation*}
\dot z = A_c z + B_c v, \qquad y = C_c z,
\end{equation*}
где
\begin{equation*}
A_c = \begin{pmatrix}0 & 1\\ 0 & 0\end{pmatrix},\quad B_c = \begin{pmatrix}0\\1\end{pmatrix},\quad C_c = \begin{pmatrix}1 & 0\end{pmatrix}.
\end{equation*}
Нуль-динамика задаётся внутренними координатами при $z\equiv 0$ и имеет вид $\dot \eta = f_0(\eta,0)$, что соответствует внутренним уравнениям системы при $x_1\equiv 0$, $x_2\equiv 0$.

Общие формулы метода линеаризации обратной связью для относительной степени $r$:
\begin{equation*}
\gamma(x) = L_g L_f^{r-1} h(x), \qquad \alpha(x) = -\,\frac{L_f^{r} h(x)}{L_g L_f^{r-1} h(x)}.
\end{equation*}
В нашем случае $r=2$, поэтому
\begin{equation*}
\gamma(x) = L_g L_f h(x) = 1, \qquad \alpha(x) = -\,L_f^{2} h(x) = -(2x_1 - 2x_2 - x_1 x_3),
\end{equation*}
и закон управления принимает вид $u = \alpha(x) + \gamma(x)^{-1} v = v - 2x_1 + 2x_2 + x_1 x_3$ (совпадает с полученным выше).

Замечание: внешняя составляющая $\xi = (z_1, z_2)^T$ соответствует канонической цепочке интеграторов, а внутренняя координата $\eta = x_3$ описывает нуль-динамику.

\subsection*{Полная форма в координатах $z=(\eta, \xi_1, \xi_2)$}
Введём преобразование
\begin{equation*}
\eta = x_3, \qquad \xi_1 = z_1 = x_1, \qquad \xi_2 = z_2 = -x_1 + x_2.
\end{equation*}
Отсюда $x_1 = \xi_1$, $x_2 = \xi_1 + \xi_2$, $x_3 = \eta$. Динамика системы в этих координатах:
\begin{align*}
\dot\eta &= x_1 + x_1 x_2 - 2 x_3 = \xi_1 + \xi_1^2 + \xi_1 \xi_2 - 2\eta, \\
\dot\xi_1 &= \xi_2, \\
\dot\xi_2 &= L_f^2 h + L_g L_f h\, u = (-2\xi_2 - \xi_1\eta) + u.
\end{align*}
Выбирая
\begin{equation*}
u = \alpha(x) + v = (2\xi_2 + \xi_1 \eta) + v,
\end{equation*}
получаем полностью линеаризованную внешнюю динамику
\begin{equation*}
\dot\xi_1 = \xi_2, \qquad \dot\xi_2 = v \;\; \Longleftrightarrow \;\; \dot\xi = A_c\xi + B_c v,
\end{equation*}
а внутренняя (нуль-) динамика в явном виде
\begin{equation*}
\dot\eta = \xi_1 + \xi_1^2 + \xi_1 \xi_2 - 2\eta, \qquad \text{и при } \xi\equiv 0: \; \dot\eta = -2\eta.
\end{equation*}
Для стабилизации внешней части берём $v = -K\xi = -k_1\xi_1 - k_2\xi_2$ (получается $\dot\xi = (A_c - B_cK)\xi$), а итоговый закон управления в исходных переменных:
\begin{equation*}
u(x) = 2(-x_1 + x_2) + x_1 x_3 - k_1 x_1 - k_2(-x_1 + x_2).
\end{equation*}

Выбирая $v = -k_1 z_1 - k_2 z_2$ с $k_1,k_2>0$, получаем желаемые собственные значения замкнутой внешней части, а вся система стабилизируется (минимально-фазовость проверена выше).

\paragraph{О полной линеаризации.}
Заметим, что для выбранного выхода $y=h(x)=x_1$ относительная степень равна $r=2< n=3$. Это означает, что статической обратной связью и диффеоморфизмом нельзя привести всю систему к канонической линейной форме цепочки трёх интеграторов (полная линеаризация невозможна), поскольку присутствует ненулевая внутренняя динамика. В нашем случае нуль-динамика соответствует координате $\eta=x_3$ и имеет линейный устойчивый вид $\dot x_3 = -2 x_3$ при $z\equiv 0$.

Если требуется именно \emph{полная} линеаризация, возможны два пути: (i) подобрать другой выход $y=\tilde h(x)$ с относительной степенью $r=3$ (для данной структуры в окрестности изучаемой рабочей точки такой выход не существует), либо (ii) выполнить динамическое расширение (ввести динамику привода, например $\dot u = w$), что повышает относительную степень до $r=3$ для расширенной системы и позволяет привести её к полной линейной форме Брунoвского по переменным $(z, u)$.

\subsection*{Моделирование управляемой системы}

\begin{figure}[H]
\centering
\includegraphics[width=0.9\textwidth]{task2/feedback_linearization.png}
\caption{Сравнение управляемой и неуправляемой систем}
\label{fig:feedback_linearization}
\end{figure}

Результаты моделирования показывают:
\begin{itemize}
\item Управляемая система экспоненциально сходится к началу координат
\item Неуправляемая система остается неустойчивой
\item Закон управления обеспечивает глобальную стабилизацию
\end{itemize}

\subsection*{Результаты задачи 2}

\textbf{Закон управления:} $u = -x_1 - x_2 + x_1 x_3$

\textbf{Относительная степень:} $r = 2$

\textbf{Стабилизация:} Глобальная стабилизация начала координат достигнута

\section*{Заключение}

В данной лабораторной работе были рассмотрены методы линеаризации обратной связью для нелинейных систем управления. Выполнены следующие задачи:

\begin{enumerate}
\item \textbf{Анализ линеаризуемости по входу-выходу:} для первой системы установлена линеаризуемость с относительной степенью $r = 1$ и минимально-фазовость.

\item \textbf{Преобразование в нормальную форму:} получены координаты нормальной формы с областью определения $\mathbb{R}^3$.

\item \textbf{Синтез закона управления:} для второй системы синтезирован закон управления $u = -x_1 - x_2 + x_1 x_3$, обеспечивающий глобальную стабилизацию начала координат.

\item \textbf{Численное моделирование:} подтверждена эффективность синтезированных законов управления.
\end{enumerate}

Работа продемонстрировала эффективность применения методов линеаризации обратной связью к практическим задачам управления нелинейными системами. Все поставленные задачи решены с использованием численного моделирования и визуализации результатов.
