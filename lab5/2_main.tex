\section*{Введение}

В данной лабораторной работе рассматривается синтез стабилизирующих регуляторов на основе скользящих режимов для нелинейных систем управления. Метод скользящих режимов обеспечивает робастность к неопределенностям параметров и внешним возмущениям за счет использования разрывных или непрерывных законов управления с насыщением.

Основные задачи работы:
\begin{enumerate}
\item Синтез разрывного и непрерывного регуляторов для системы с параметрическими неопределенностями
\item Синтез стабилизирующего регулятора для системы с неизвестными параметрами
\item Разработка непрерывного регулятора для стабилизации маятника
\end{enumerate}

Для каждой системы предполагается, что весь вектор состояния измерим, и необходимо провести анализ устойчивости и математическое моделирование синтезированных регуляторов.

\section*{Задача 1. Синтез разрывного и непрерывного регуляторов}

Рассмотрим систему:
\begin{align}
\dot{x}_1 &= x_2 + \sin x_1 \\
\dot{x}_2 &= \theta_1 x_1^2 + (2 + \theta_2)u
\end{align}

где параметры $\theta_1$ и $\theta_2$ неизвестны, но ограничены: $|\theta_1| \leq 1$, $|\theta_2| \leq 1$.

\subsection*{Выбор поверхности скольжения}

Выбираем поверхность скольжения в виде:
\begin{equation}
s = c_1 x_1 + x_2 = 0
\end{equation}

где $c_1 > 0$ --- параметр настройки. На поверхности скольжения система имеет динамику:
\begin{equation}
\dot{x}_1 = -c_1 x_1 + \sin x_1
\end{equation}

При малых $x_1$ и выборе $c_1 > 1$ эта динамика асимптотически устойчива.

\subsection*{Производная поверхности скольжения}

Вычисляем производную поверхности скольжения:
\begin{equation}
\dot{s} = c_1 \dot{x}_1 + \dot{x}_2 = c_1(x_2 + \sin x_1) + \theta_1 x_1^2 + (2 + \theta_2)u
\end{equation}

Для обеспечения условия притяжения к поверхности скольжения $\dot{s} = -k s$ (где $k > 0$) получаем:
\begin{equation}
(2 + \theta_2)u = -k s - c_1(x_2 + \sin x_1) - \theta_1 x_1^2
\end{equation}

\subsection*{Разрывный регулятор}

Разрывный регулятор имеет вид:
\begin{equation}
u = u_{eq} - M \cdot \text{sign}(s)
\end{equation}

где $u_{eq}$ --- эквивалентное управление (номинальное, при $\theta_1 = 0$, $\theta_2 = 0$):
\begin{equation}
u_{eq} = -\frac{k s + c_1(x_2 + \sin x_1)}{2}
\end{equation}

где $k > 0$ --- коэффициент, обеспечивающий $\dot{s} = -k s$ на поверхности скольжения.

а $M > 0$ выбирается достаточно большим для компенсации неопределенностей:
\begin{equation}
M \geq \frac{|\theta_1| x_1^2 + |\theta_2| |u_{eq}| + \delta}{2 - |\theta_2|}
\end{equation}

где $\delta > 0$ --- запас робастности.

Условие притяжения к поверхности скольжения:
\begin{equation}
s \dot{s} = s(-k s - \Delta) = -k s^2 - s \Delta < 0
\end{equation}

где $\Delta$ --- неопределенность, компенсируемая разрывной частью управления.

\subsection*{Непрерывный регулятор}

Для устранения явления ``дрожания'' (chattering) используется непрерывный регулятор с функцией насыщения:
\begin{equation}
u = u_{eq} - M \cdot \tanh\left(\frac{s}{\phi}\right)
\end{equation}

где $\phi > 0$ --- ширина граничного слоя. Функция $\tanh(s/\phi)$ аппроксимирует $\text{sign}(s)$ и обеспечивает непрерывность управления.

В граничном слое $|s| < \phi$ система находится в режиме ``квазискольжения'', а вне граничного слоя поведение близко к разрывному регулятору.

\subsection*{Анализ устойчивости}

Рассмотрим функцию Ляпунова:
\begin{equation}
V = \frac{1}{2}s^2
\end{equation}

Производная функции Ляпунова:
\begin{equation}
\dot{V} = s \dot{s} = s(-k s - \Delta + (2 + \theta_2)u_{sw})
\end{equation}

где $u_{sw} = -M \cdot \tanh(s/\phi)$ --- разрывная часть управления.

При правильном выборе $M$ обеспечивается $\dot{V} < 0$ вне малой окрестности $s = 0$, что гарантирует притяжение к поверхности скольжения и последующую стабилизацию в начало координат.

\subsection*{Моделирование системы 1}

\begin{figure}[H]
\centering
\includegraphics[width=\textwidth]{task1_sliding_mode.png}
\caption{Моделирование системы 1 с разрывным и непрерывным регуляторами}
\label{fig:task1_sliding_mode}
\end{figure}

На рисунке~\ref{fig:task1_sliding_mode} представлены результаты моделирования системы с разрывным и непрерывным регуляторами. Видно, что оба регулятора обеспечивают стабилизацию системы в начало координат. Непрерывный регулятор устраняет явление ``дрожания'', характерное для разрывного регулятора, при сохранении робастных свойств.

\section*{Задача 2. Синтез регулятора для системы с неизвестными параметрами}

Рассмотрим систему:
\begin{align}
\dot{x}_1 &= x_2 + a_1 x_1 \sin x_1 \\
\dot{x}_2 &= a_2 x_1 x_2 + 3u
\end{align}

где параметры $a_1$ и $a_2$ неизвестны, но ограничены: $|a_1 - 1| \leq 1$, $|a_2 - 1| \leq 1$, то есть $a_1 \in [0, 2]$, $a_2 \in [0, 2]$.

\subsection*{Выбор поверхности скольжения}

Выбираем поверхность скольжения:
\begin{equation}
s = c_1 x_1 + x_2 = 0
\end{equation}

где $c_1 > 0$ --- параметр настройки.

\subsection*{Производная поверхности скольжения}

Вычисляем производную:
\begin{equation}
\dot{s} = c_1(x_2 + a_1 x_1 \sin x_1) + a_2 x_1 x_2 + 3u
\end{equation}

Для обеспечения $\dot{s} = -k s$ получаем:
\begin{equation}
3u = -k s - c_1(x_2 + a_1 x_1 \sin x_1) - a_2 x_1 x_2
\end{equation}

\subsection*{Синтез регулятора}

Эквивалентное управление (номинальное, при $a_1 = 1$, $a_2 = 1$):
\begin{equation}
u_{eq} = -\frac{c_1(x_2 + x_1 \sin x_1) + x_1 x_2}{3}
\end{equation}

Непрерывный регулятор:
\begin{equation}
u = u_{eq} - M \cdot \tanh\left(\frac{s}{\phi}\right)
\end{equation}

где $M > 0$ выбирается для компенсации неопределенностей параметров $a_1$ и $a_2$.

\subsection*{Анализ устойчивости}

Рассмотрим функцию Ляпунова $V = \frac{1}{2}s^2$. Производная:
\begin{equation}
\dot{V} = s \dot{s} = s(-k s - \Delta_{a_1} - \Delta_{a_2} + 3u_{sw})
\end{equation}

где $\Delta_{a_1}$ и $\Delta_{a_2}$ --- неопределенности, связанные с параметрами $a_1$ и $a_2$.

При выборе $M$ достаточного большого обеспечивается $\dot{V} < 0$, что гарантирует притяжение к поверхности скольжения и стабилизацию системы.

\subsection*{Моделирование системы 2}

\begin{figure}[H]
\centering
\includegraphics[width=\textwidth]{task2_sliding_mode.png}
\caption{Моделирование системы 2 с регулятором для различных значений параметров}
\label{fig:task2_sliding_mode}
\end{figure}

На рисунке~\ref{fig:task2_sliding_mode} представлены результаты моделирования системы с регулятором для различных комбинаций параметров $a_1$ и $a_2$ в допустимых пределах. Видно, что регулятор обеспечивает стабилизацию системы независимо от конкретных значений параметров, демонстрируя робастность метода скользящих режимов.

\section*{Задача 3. Стабилизация маятника}

Рассмотрим уравнение движения маятника:
\begin{equation}
ml\ddot{\theta} + mg \sin \theta + kl\dot{\theta} = \frac{T}{l} + mh(t)\cos\theta
\end{equation}

где:
\begin{itemize}
\item $l$ --- длина маятника, $0.8 \leq l \leq 1$ м
\item $m$ --- масса, $0.5 \leq m \leq 1$ кг
\item $k$ --- коэффициент трения, $0.1 \leq k \leq 0.2$ Н·с/м
\item $h(t)$ --- горизонтальное возмущение, $|h(t)| \leq 0.5$ м/с²
\item $T$ --- управляющий момент
\item $g = 9.81$ м/с² --- ускорение свободного падения
\end{itemize}

Требуется стабилизировать маятник при $\theta = 0$ для произвольных начальных условий.

\subsection*{Приведение к нормальной форме}

Вводя переменные состояния $x_1 = \theta$, $x_2 = \dot{\theta}$, получаем систему:
\begin{align}
\dot{x}_1 &= x_2 \\
\dot{x}_2 &= -\frac{g}{l}\sin x_1 - k x_2 + \frac{T}{ml^2} + \frac{h(t)}{l}\cos x_1
\end{align}

\subsection*{Выбор поверхности скольжения}

Выбираем поверхность скольжения:
\begin{equation}
s = c_1 x_1 + x_2 = 0
\end{equation}

где $c_1 > 0$ --- параметр настройки. На поверхности скольжения динамика угла:
\begin{equation}
\dot{x}_1 = -c_1 x_1 + \text{малые члены}
\end{equation}

что обеспечивает асимптотическую стабилизацию при $c_1 > 0$.

\subsection*{Производная поверхности скольжения}

Вычисляем производную:
\begin{equation}
\dot{s} = c_1 x_2 - \frac{g}{l}\sin x_1 - k x_2 + \frac{T}{ml^2} + \frac{h(t)}{l}\cos x_1
\end{equation}

Для обеспечения $\dot{s} = -k_{sliding} s$ (где $k_{sliding} > 0$) получаем:
\begin{equation}
\frac{T}{ml^2} = -k_{sliding} s - c_1 x_2 + \frac{g}{l}\sin x_1 + k x_2 - \frac{h(t)}{l}\cos x_1
\end{equation}

\subsection*{Синтез непрерывного регулятора}

Эквивалентное управление (номинальное, при средних значениях параметров):
\begin{equation}
T_{eq} = ml^2\left(c_1 x_2 - \frac{g}{l}\sin x_1 + k x_2\right)
\end{equation}

Непрерывный регулятор:
\begin{equation}
T = T_{eq} - M \cdot \tanh\left(\frac{s}{\phi}\right)
\end{equation}

где $M > 0$ выбирается для компенсации неопределенностей параметров $l$, $m$, $k$ и возмущения $h(t)$.

\subsection*{Анализ устойчивости}

Рассмотрим функцию Ляпунова $V = \frac{1}{2}s^2$. Производная:
\begin{equation}
\dot{V} = s \dot{s} = s(-k_{sliding} s - \Delta_{params} - \Delta_h + \frac{T_{sw}}{ml^2})
\end{equation}

где $\Delta_{params}$ --- неопределенности параметров, $\Delta_h$ --- возмущение от $h(t)$, $T_{sw} = -M \cdot \tanh(s/\phi)$ --- разрывная часть управления.

При правильном выборе $M$ обеспечивается $\dot{V} < 0$, что гарантирует притяжение к поверхности скольжения и стабилизацию маятника в вертикальное положение.

\subsection*{Моделирование системы 3}

\begin{figure}[H]
\centering
\includegraphics[width=\textwidth]{task3_pendulum.png}
\caption{Моделирование стабилизации маятника с регулятором для различных параметров}
\label{fig:task3_pendulum}
\end{figure}

На рисунке~\ref{fig:task3_pendulum} представлены результаты моделирования стабилизации маятника с регулятором для различных комбинаций параметров в допустимых пределах и различных возмущений $h(t)$. Видно, что регулятор обеспечивает стабилизацию маятника в вертикальное положение ($\theta = 0$) независимо от начальных условий, параметров системы и внешних возмущений.

\section*{Заключение}

В ходе выполнения лабораторной работы были синтезированы стабилизирующие регуляторы на основе скользящих режимов для трех различных нелинейных систем:

\begin{enumerate}
\item Для системы с параметрическими неопределенностями синтезированы разрывный и непрерывный регуляторы. Непрерывный регулятор устраняет явление ``дрожания'', сохраняя робастные свойства метода.

\item Для системы с неизвестными параметрами синтезирован непрерывный регулятор, демонстрирующий робастность к вариациям параметров в заданных пределах.

\item Для маятника синтезирован непрерывный регулятор, обеспечивающий стабилизацию в вертикальное положение при наличии неопределенностей параметров и внешних возмущений.
\end{enumerate}

Все синтезированные регуляторы продемонстрировали эффективность метода скользящих режимов для обеспечения робастной стабилизации нелинейных систем с неопределенностями.

